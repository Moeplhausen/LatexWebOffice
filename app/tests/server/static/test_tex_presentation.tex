\documentclass{beamer}
%bla
\usepackage[utf8]{inputenc}
\usepackage{default}
\usetheme{Warsaw}
\begin{document}

  \begin{frame}
    \frametitle{Allgemein}
    \begin{block}{Tests verteilt auf Kategorien}
     \begin{itemize}
      \item Allgemein
      \item Projekte  
      \item Dateien 
      \item Ordner
     \end{itemize}

    \end{block}
    
    \begin{block}{Tests}
    \begin{itemize}
     \item Jeder einzelner ``command'' getestet
     \item Fokus auf Sicherheit
     \item Relativ umfangreiche Tests ($\approx1500$ Zeilen nur für Tests)
     \begin{itemize}
      \item Nicht alles abgedeckt
     \end{itemize}
     \item Wenig Seiteneffekte
    \end{itemize}
    \end{block}

  \end{frame}

  
  \begin{frame}
   \frametitle{Gemeinsamkeiten der Tests}
   \begin{block}{Abgefangen in der Execute-Verteilmethode}
    \begin{itemize}
     \item Nicht eingeloggt
     \item Kein POST request
     \item Zahl wurde erwartet, aber Zeichenkette bekommen
     \item Es wurden nicht alle erwarteten Parameter eines 'commands' übergeben
    \end{itemize}
   \end{block}
   \begin{block}{In den einzelnen Methoden}
    \begin{itemize}
     \item Erfolgsmeldung bei einer richtigen Eingabe
     \item Überprüfen, ob der User die benötigten Rechte für eine bestimmte Operation hat
     \begin{itemize}
      \item Durch allgemeine Hilfsmethoden erleichtert
      \item Schwierig, Rechte für beliebige Operationen in einer einzigen Methode zu kontrollieren
     \end{itemize}
    \end{itemize}
   \end{block}
  \end{frame}
  
    \begin{frame}
    \frametitle{Sicherheit}
    \begin{block}{Was wurde getestet}
    \begin{itemize}
     \item Kein User darf Projekte, die ihm nicht gehören, auslesen oder verändern 
     \begin{itemize}
      \item Immer die gleiche Fehlermeldung 'NOTENOUGHRIGHTS'
      \item Keine Änderung in der Datenbank
      \begin{itemize}
       \item Nicht konsequent getestet
      \end{itemize}
     \end{itemize}
    \end{itemize}
    \end{block}
  \end{frame}
  
  
  \begin{frame}
   \frametitle{Beispiel Projectcreate}
   \begin{block}{Gewünschtes Verhalten}
    \begin{itemize}
     \item Projekt wird bei Angabe eines Namens (nicht leer, usw) erstellt
     \begin{itemize}
      \item Es existiert ein (root) Verzeichnis, welches zu dem Projekt gehört
      \item eine main.tex Datei in diesem Verzeichnis
     \end{itemize}
     \item Projekt wird nicht erstellt, 
     \begin{itemize}
     \item falls ein Projekt mit gleichen Namen vom gleichen User bereits existiert
     \item falls der gewünschte Name vom Projekt leer oder ungültige Zeichen enthält
     \end{itemize}
     \item Das Projekt gehört am Ende wirklich dem richtigen User
     \item Richtige Antwort vom Server an Client für jeden Fall
    \end{itemize}
   \end{block}
  \end{frame}


% etc
\end{document}